\documentclass[a4paper,12pt,article]{memoir}

\setlrmarginsandblock{1cm}{*}{1}
\setulmarginsandblock{2cm}{*}{1}
\checkandfixthelayout[nearest]
\setlength{\parindent}{0pt}

\usepackage[utf8]{inputenc}
\usepackage[norwegian]{babel}
\usepackage[T1]{fontenc}
\usepackage{lmodern}
\usepackage{amsmath, amssymb, bm, mathtools, mathdots}
\usepackage[normalem]{ulem}
\usepackage{array, booktabs, tabularx}
\usepackage{graphicx, caption, subfig, xcolor}
\usepackage{enumerate}
\usepackage{enumitem}
\usepackage{hyperref}
\usepackage{listings}

\captionsetup{font=small,labelfont=bf}

\makepagestyle{mypagestyle}
\copypagestyle{mypagestyle}{empty}
\makeoddhead{mypagestyle}{TDT4310, Spring 2022\\Lab date: January 18, 2022}{\quad}{Prof. Björn Gambäck\\TA. Tollef Jørgensen\\}
\makeheadrule{mypagestyle}{\textwidth}{\normalrulethickness}
\makeoddfoot{mypagestyle}{}{\thepage~of~\thelastpage}{} 

\pagestyle{mypagestyle}

\definecolor{codegreen}{rgb}{0,0.6,0}
\definecolor{codegray}{rgb}{0.5,0.5,0.5}
\definecolor{text}{rgb}{0.58,0,0.82}
\definecolor{backcolour}{rgb}{0.95,0.95,0.92}

\lstdefinestyle{code}{
    backgroundcolor=\color{backcolour},   
    commentstyle=\color{codegreen},
    keywordstyle=\color{magenta},
    numberstyle=\tiny\color{codegray},
    stringstyle=\color{text},
    basicstyle=\ttfamily\footnotesize,
    breakatwhitespace=true,         
    breaklines=true,                 
    captionpos=b,                    
    keepspaces=true,                 
    numbers=left,                    
    numbersep=5pt,                  
    showspaces=false,                
    showstringspaces=false,
    showtabs=false,                  
    tabsize=4
}
\lstset{style=code}
%\lstset{escapeinside={<}{>}}

\begin{document}

\begin{center}
\Large{\textbf{Lab 1 -- Corpora and Text Processing}}
\\
\large{\textbf{Name Nameson, your-ntnu-username}}
\end{center}

\section*{Instructions}
Here you could write some instructions required for running the code if necessary, such as modules used etc.
\section*{Exercises}
\subsection{1 - Some exercise here}
this is the layout for tasks
\subsubsection{A subsubsection}
categorize on even smaller sections within exercises.
Code can be written using lstlisting:
\begin{lstlisting}[language=Python]
def times_five(x):
    return x * 5
\end{lstlisting}

\begin{itemize}
    \item this is a normal list
    \item this is a normal list
    \item this is a normal list
\end{itemize}

\begin{enumerate}\itemsep0em
    \item this is a compact list with enumeration
    \item this is a compact list with enumeration
    \item this is a compact list with enumeration
\end{enumerate}

\begin{enumerate}[label=(\alph*)]\itemsep0em
    \item this is a compact list with a, b, c
    \item this is a compact list with a, b, c
    \item this is a compact list with a, b, c
\end{enumerate}

\end{document}